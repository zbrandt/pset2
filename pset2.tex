\documentclass{article}

\usepackage{fancyhdr}
\usepackage{extramarks}
\usepackage{amsmath}
\usepackage{amsthm}
\usepackage{amsfonts}
\usepackage{tikz}
\usepackage[plain]{algorithm}
\usepackage{algpseudocode}
\usepackage{enumitem}

\usetikzlibrary{automata,positioning}

%
% Basic Document Settings
%

\topmargin=-0.45in
\evensidemargin=0in
\oddsidemargin=0in
\textwidth=6.5in
\textheight=9.0in
\headsep=0.25in

\linespread{1.1}

\pagestyle{fancy}
\lhead{\hmwkAuthorName}
\chead{\hmwkClass\ (\hmwkClassInstructor\ \hmwkClassTime): \hmwkTitle}
\rhead{\firstxmark}
\lfoot{\lastxmark}
\cfoot{\thepage}

\renewcommand\headrulewidth{0.4pt}
\renewcommand\footrulewidth{0.4pt}

\setlength\parindent{0pt}

%
% Create Problem Sections
%

\newcommand{\enterProblemHeader}[1]{
    \nobreak\extramarks{}{Question \arabic{#1} continued on next page\ldots}\nobreak{}
    \nobreak\extramarks{Question \arabic{#1}}{Question \arabic{#1} continued on next page\ldots}\nobreak{}
}

\newcommand{\exitProblemHeader}[1]{
    \nobreak\extramarks{Question \arabic{#1}}{Question \arabic{#1} continued on next page\ldots}\nobreak{}
    \stepcounter{#1}
    \nobreak\extramarks{Question \arabic{#1}}{}\nobreak{}
}

\setcounter{secnumdepth}{0}
\newcounter{partCounter}
\newcounter{homeworkProblemCounter}
\setcounter{homeworkProblemCounter}{1}
\nobreak\extramarks{Question \arabic{homeworkProblemCounter}}{}\nobreak{}

%
% Homework Problem Environment
%
% This environment takes an optional argument. When given, it will adjust the
% problem counter. This is useful for when the problems given for your
% assignment aren't sequential. See the last 3 problems of this template for an
% example.
%
\newenvironment{homeworkProblem}[1][-1]{
    \ifnum#1>0
        \setcounter{homeworkProblemCounter}{#1}
    \fi
    \section{Question \arabic{homeworkProblemCounter}}
    \setcounter{partCounter}{1}
    \enterProblemHeader{homeworkProblemCounter}
}{
    \exitProblemHeader{homeworkProblemCounter}
}

%
% Homework Details
%   - Title
%   - Due date
%   - Class
%   - Section/Time
%   - Instructor
%   - Author
%

\newcommand{\hmwkTitle}{Problem Set\ \#2}
\newcommand{\hmwkDueDate}{April 5, 2024}
\newcommand{\hmwkClass}{Econometrics}
\newcommand{\hmwkClassTime}{Section 101}
\newcommand{\hmwkClassInstructor}{Professor Ben Faber}
\newcommand{\hmwkAuthorName}{\textbf{Zachary Brandt}}

%
% Title Page
%

\title{
    \vspace{2in}
    \textmd{\textbf{\hmwkClass:\ \hmwkTitle}}\\
    \normalsize\vspace{0.1in}\small{Due\ on\ \hmwkDueDate\ at 4:00pm}\\
    \vspace{0.1in}\large{\textit{\hmwkClassInstructor\ \hmwkClassTime}}
    \vspace{3in}
}

\author{\hmwkAuthorName}
\date{}

\renewcommand{\part}[1]{\textbf{\large Part \Alph{partCounter}}\stepcounter{partCounter}\\}

%
% Various Helper Commands
%

% Useful for algorithms
\newcommand{\alg}[1]{\textsc{\bfseries \footnotesize #1}}

% For derivatives
\newcommand{\deriv}[1]{\frac{\mathrm{d}}{\mathrm{d}x} (#1)}

% For partial derivatives
\newcommand{\pderiv}[2]{\frac{\partial}{\partial #1} (#2)}

% Integral dx
\newcommand{\dx}{\mathrm{d}x}

% Alias for the Solution section header
\newcommand{\solution}{\textbf{\large Solution}}

% Probability commands: Expectation, Variance, Covariance, Bias
\newcommand{\E}{\mathrm{E}}
\newcommand{\Var}{\mathrm{Var}}
\newcommand{\Cov}{\mathrm{Cov}}
\newcommand{\Bias}{\mathrm{Bias}}

\begin{document}

\maketitle

\pagebreak

\begin{homeworkProblem}

    The Ministry of Commerce in a large country wants to know the causal effect of membership in a
    local Chamber of Commerce on firm revenues and profits. Firms pay for their membership and they
    supposedly benefit from the network of information and contacts that the local Chambers of
    Commerce offer them. But usually, it is only a small minority of all firms that end up paying for the
    membership. \newline

    The Ministry plans to estimate the causal effect of being a member in a local Chamber of Commerce
    by letting the Ministry’s staff estimate the average percentage change in annual firm sales between
    firms that are members and the rest of the firms that are non-members of their local Chamber of
    Commerce. 

    \begin{enumerate}
        \item[A)] Write down the OLS regression specification that the Ministry’s staff could use to implement their analysis described above. Interpret what the intercept and slope coefficients would capture in such a specification.
        \item[B)] Using notation from the Potential Outcomes Framework, briefly explain the concept of the Average Treatment Effect (ATE) to the Minister, and how what they plan to estimate in A) relates to this definition.
        \item[C)] Referring to the expressions you use in your answer to B), explain why a randomized control trial (RCT) could be useful, and very briefly describe the basics of the RCT design for how the Ministry could set this up.
        \item[D)] The Ministry mentions that it has no legal authority to force firms to become members in their local Chambers of Commerce. Using notation from the Potential Outcomes Framework, explain why this information could be important for the interpretation of the results from the RCT relative to the ATE, and how the Ministry should address this concern in the RCT analysis?
        \item[E)] The Ministry talked to other economists, and now it is worried about spillover effects on the control group. The staff don’t fully understand what the concern is, however. Briefly explain to them the intuition behind this concern, and explain how they could potentially address it when designing the RCT.
    \end{enumerate}
    
    \pagebreak

    \part

    Write down the OLS regression specification that the Ministry’s staff could use to implement their analysis described above. Interpret what the intercept and slope coefficients would capture in such a specification.
    \\
    
    \solution  

    The following OLS regression specification is one that the Ministry could implement:

    \[ 
        ln(Y_i) = \beta_0 + \beta_1 D_i + u_i 
    \]
    
    where
        \begin{itemize}[label={}, itemsep=0pt]
            \item the subscript $i$ runs over the observations, $i = 1,\dots,n$;
            \item $Y_i$ is the \textit{dependent variable}, annual firm sales 
            \item $D_i$ is the \textit{dummy variable}, $D_i=1$ if the firm is a local Chamber of Commerce member and $0$ otherwise 
            \item $\beta_0$ is the \textit{intercept} of the regression, the population mean value of annual, non-member firm sales 
            \item $\beta_1$ is the coefficient on $D_i$, associating a change in $D_i$ by one unit with a $100\beta_1\%$ change in $Y_i$
            \item $\beta_0 + \beta_1$ is the population mean value of annual, member firm sales
            \item $u_i$ is the \textit{error} term, all the factors responsiblze for the difference between predicted and observed values
        \end{itemize}

    \pagebreak


    \part
    
    Using notation from the Potential Outcomes Framework, briefly explain the concept of the Average Treatment Effect (ATE) to the Minister, and how what they plan to estimate in A) relates to this definition.
    \\

    \solution
    
    One core challenge with evaluating the causal effect of membership in a local Chamber of Commerce on firm revenues and profits is that we cannot observe a firm $i$ in two different states of the world: one where the firm is a member, and one where it is not. Instead, we can compare the mean outcomes of two firm groups (members vs. non-members) to learn about the true, average treatment effect (ATE) of membership.  \\
    \\
    The ATE is defined as:
    \[
        ATE = E(Y_i(1) - Y_i(0))
    \]
    where $Y_i(1)$ represents the potential outcome (the annual sales) of a firm $i$ if part of a local Chamber of Commerce and $Y_i(0)$ represents the annual sales for the \textbf{same} firm if not. \\
    \\
    To compare the mean outcomes of the two groups of firms (members vs. non-members of a local Chamber of Commerce), we instead estimate:
    \[
        E(Y_i(1) \:|\: X_i = 1) - E(Y_i(0) \:|\: X_i = 0)  
    \]
    where we condition our observation of $Y_i$ on $X_i$, with $X_i = 1$ representing if firm $i$ is part of a local Chamber of Commerce, and $X_i = 0$ if it is not. The above is not necessarily equal to the ATE defined earlier because $X_i$, or membership status, is \textit{not independent} of potential outcomes (annual firm sales). We would have to show first that, in the absence of a local Chamber of Commerce, member and non-member firms would have to be on average identical (there would have to be no selection bias).\\
    \\
    Unpacking the above expression we can see the effect of selection bias:
    \[
        \begin{split}
            E(Y_i(1) \:|\: D_i = 1) - E(Y_i(0) \:|\: D_i = 1) &= \underbrace{E(Y_i(1) \:|\: D_i = 1) - E(Y_i(0) \:|\: D_i = 1)}_\text{Average Treatment Effect on the Treated}
            \\
            &+ \underbrace{E(Y_i(0) \:|\: D_i = 1) - E(Y_i(0) \:|\: D_i = 0)}_\text{Selection Bias} 
        \end{split}
    \]
    The first term is the average treatment effect on the treated (ATT), the difference between $Y_i(1)$ and $Y_i(0)$ for the group of firms that are members of a local Chamber of Commerce. 
    The second term estimates the selection bias by comparing $Y_i(0)$, annual sales before being a member, between members, $X_i = 1$, and non-members, $X_i = 0$. $E(Y_i(0) \:|\: X_i = 1)$ cannot be observed directly for both of these terms, because we cannot rerun the experiment a second time where the treatment is not applied.
    The bias term appears because firms that are members of a local Chamber of Commerce may have different outcomes than firms that are not members, \textit{even in the absence of a local Chamber of Commerce}.
    \pagebreak


    \part 
    
    Referring to the expressions you use in your answer to B), explain why a randomized control trial (RCT) could be useful, and very briefly describe the basics of the RCT design for how the Ministry could set this up.
    \\

    \solution

    In a RCT, treatment and control groups are randomly selected, and the average outcomes of these two groups are compared after the treatment. By randomly allocating treatment status, we can use the simple difference in outcomes between treatment and control groups to estimate the ATE. This is because the randomization removes selection bias by ensuring that member firms and non-member firms are comparable in terms of both observable and unobservable characteristics and that any difference between the two groups is not due to systematic differences. This implies that the ATE = ATT. \\
    \\
    Connecting back to the potential outcomes framework, $D_i = 1$ indicates if firm $i$ is a member of a local Chamber of Commerce (the treatment), and $D_i = 0$ if the firm is not ($D_i$ indicates treatment status). $Y_i(1)$ are the annual sales (the potential outcome) for a firm $i$ if treated (part of a local Chamber or Commerce), and $Y_i(0)$ are the annual sales for the same firm if not. \\
    \\
    The ATE is then defined as:
    \[
        \begin{split}
            ATE &= E(Y_i(1) - Y_i(0))
        \end{split}
    \]
    
    If the treatment assignment $D_i$ is assigned randomly and independent of the potential outcomes of the firms ($D_i\perp (Y_i(1), Y_i(0))$), we can compare the mean outcomes of the two groups (members vs. non-members of a local Chamber of Commerce):
    \[
        \begin{split}
            E(Y_i(1) \:|\: D_i = 1) - E(Y_i(0) \:|\: D_i = 1) &= \underbrace{E(Y_i(1) \:|\: D_i = 1) - E(Y_i(0) \:|\: D_i = 1)}_\text{Average Treatment Effect on the Treated}
            \\
            &+ \underbrace{E(Y_i(0) \:|\: D_i = 1) - E(Y_i(0) \:|\: D_i = 0)}_\text{Selection Bias} 
            \\
            &= E(Y_i(1) \:|\: D_i = 1) - E(Y_i(0) \:|\: D_i = 1) \: \: \: \text{(selection bias is zero)}
            \\
            &= ATT
        \end{split}
    \]
    because the treatment status is independent of outcomes, we can also show: 
    \[
        \begin{split}
            ATT &= E(Y_i(1) - Y_i(0)) = ATE
        \end{split}
    \]


    This means that if we randomly assign treatment status (membership in a local Chamber of Commerce) and control status (non-membership), we can estimate the ATE by comparing mean sales outcomes across the two groups of observation. We can report the estimated treatment effect of the RCT design from the OLS regression, $ln(Y_i) = \beta_0 + \beta_1 D_i + u_i$. The $\beta_1$ coefficient is the difference in means between the treatment group relative to the control group. Dividing this difference by its standard error produces a $t$-statistic that allows us to determine whether the observed difference in annual sales between firms part of a local Chamber of Commerce and firms that are not is statistically significant.   
    \pagebreak

    \part 
    
    The Ministry mentions that it has no legal authority to force firms to become members in their local Chambers of Commerce. Using notation from the Potential Outcomes Framework, explain why this information could be important for the interpretation of the results from the RCT relative to the ATE, and how the Ministry should address this concern in the RCT analysis?
    \\

    \solution

    In the case where assigned treatment status, $D_i$, cannot be enforced on firms,  

\end{homeworkProblem}

\pagebreak

\begin{homeworkProblem}
    Suppose we would like to fit a straight line through the origin, i.e.,
    \(Y_i = \beta_1 x_i + e_i\) with \(i = 1, \ldots, n\), \(\E [e_i] = 0\),
    and \(\Var [e_i] = \sigma^2_e\) and \(\Cov[e_i, e_j] = 0, \forall i \neq
    j\).
    \\

    \part

    Find the least squares esimator for \(\hat{\beta_1}\) for the slope
    \(\beta_1\).
    \\

    \solution

    To find the least squares estimator, we should minimize our Residual Sum
    of Squares, RSS:

    \[
        \begin{split}
            RSS &= \sum_{i = 1}^{n} {(Y_i - \hat{Y_i})}^2
            \\
            &= \sum_{i = 1}^{n} {(Y_i - \hat{\beta_1} x_i)}^2
        \end{split}
    \]

    By taking the partial derivative in respect to \(\hat{\beta_1}\), we get:

    \[
        \pderiv{
            \hat{\beta_1}
        }{RSS}
        = -2 \sum_{i = 1}^{n} {x_i (Y_i - \hat{\beta_1} x_i)}
        = 0
    \]

    This gives us:

    \[
        \begin{split}
            \sum_{i = 1}^{n} {x_i (Y_i - \hat{\beta_1} x_i)}
            &= \sum_{i = 1}^{n} {x_i Y_i} - \sum_{i = 1}^{n} \hat{\beta_1} x_i^2
            \\
            &= \sum_{i = 1}^{n} {x_i Y_i} - \hat{\beta_1}\sum_{i = 1}^{n} x_i^2
        \end{split}
    \]

    Solving for \(\hat{\beta_1}\) gives the final estimator for \(\beta_1\):

    \[
        \begin{split}
            \hat{\beta_1}
            &= \frac{
                \sum {x_i Y_i}
            }{
                \sum x_i^2
            }
        \end{split}
    \]

    \pagebreak

    \part

    Calculate the bias and the variance for the estimated slope
    \(\hat{\beta_1}\).
    \\

    \solution

    For the bias, we need to calculate the expected value
    \(\E[\hat{\beta_1}]\):

    \[
        \begin{split}
            \E[\hat{\beta_1}]
            &= \E \left[ \frac{
                \sum {x_i Y_i}
            }{
                \sum x_i^2
            }\right]
            \\
            &= \frac{
                \sum {x_i \E[Y_i]}
            }{
                \sum x_i^2
            }
            \\
            &= \frac{
                \sum {x_i (\beta_1 x_i)}
            }{
                \sum x_i^2
            }
            \\
            &= \frac{
                \sum {x_i^2 \beta_1}
            }{
                \sum x_i^2
            }
            \\
            &= \beta_1 \frac{
                \sum {x_i^2 \beta_1}
            }{
                \sum x_i^2
            }
            \\
            &= \beta_1
        \end{split}
    \]

    Thus since our estimator's expected value is \(\beta_1\), we can conclude
    that the bias of our estimator is 0.
    \\

    For the variance:

    \[
        \begin{split}
            \Var[\hat{\beta_1}]
            &= \Var \left[ \frac{
                \sum {x_i Y_i}
            }{
                \sum x_i^2
            }\right]
            \\
            &=
            \frac{
                \sum {x_i^2}
            }{
                \sum x_i^2 \sum x_i^2
            } \Var[Y_i]
            \\
            &=
            \frac{
                \sum {x_i^2}
            }{
                \sum x_i^2 \sum x_i^2
            } \Var[Y_i]
            \\
            &=
            \frac{
                1
            }{
                \sum x_i^2
            } \Var[Y_i]
            \\
            &=
            \frac{
                1
            }{
                \sum x_i^2
            } \sigma^2
            \\
            &=
            \frac{
                \sigma^2
            }{
                \sum x_i^2
            }
        \end{split}
    \]

\end{homeworkProblem}

\pagebreak

\begin{homeworkProblem}
    Prove a polynomial of degree \(k\), \(a_kn^k + a_{k - 1}n^{k - 1} + \hdots
    + a_1n^1 + a_0n^0\) is a member of \(\Theta(n^k)\) where \(a_k \hdots a_0\)
    are nonnegative constants.

    \begin{proof}
        To prove that \(a_kn^k + a_{k - 1}n^{k - 1} + \hdots + a_1n^1 +
        a_0n^0\), we must show the following:

        \[
            \exists c_1 \exists c_2 \forall n \geq n_0,\ {c_1 \cdot g(n) \leq
            f(n) \leq c_2 \cdot g(n)}
        \]

        For the first inequality, it is easy to see that it holds because no
        matter what the constants are, \(n^k \leq a_kn^k + a_{k - 1}n^{k - 1} +
        \hdots + a_1n^1 + a_0n^0\) even if \(c_1 = 1\) and \(n_0 = 1\).  This
        is because \(n^k \leq c_1 \cdot a_kn^k\) for any nonnegative constant,
        \(c_1\) and \(a_k\).
        \\

        Taking the second inequality, we prove it in the following way.
        By summation, \(\sum\limits_{i=0}^k a_i\) will give us a new constant,
        \(A\). By taking this value of \(A\), we can then do the following:

        \[
            \begin{split}
                a_kn^k + a_{k - 1}n^{k - 1} + \hdots + a_1n^1 + a_0n^0 &=
                \\
                &\leq (a_k + a_{k - 1} \hdots a_1 + a_0) \cdot n^k
                \\
                &= A \cdot n^k
                \\
                &\leq c_2 \cdot n^k
            \end{split}
        \]

        where \(n_0 = 1\) and \(c_2 = A\). \(c_2\) is just a constant. Thus the
        proof is complete.
    \end{proof}
\end{homeworkProblem}

\pagebreak

%
% Non sequential homework problems
%

% Jump to problem 18
\begin{homeworkProblem}[18]
    Evaluate \(\sum_{k=1}^{5} k^2\) and \(\sum_{k=1}^{5} (k - 1)^2\).
\end{homeworkProblem}

% Continue counting to 19
\begin{homeworkProblem}
    Find the derivative of \(f(x) = x^4 + 3x^2 - 2\)
\end{homeworkProblem}

% Go back to where we left off
\begin{homeworkProblem}[6]
    Evaluate the integrals
    \(\int_0^1 (1 - x^2) \dx\)
    and
    \(\int_1^{\infty} \frac{1}{x^2} \dx\).
\end{homeworkProblem}

\end{document}
